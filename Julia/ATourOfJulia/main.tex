% !TEX program = xelatex
% vim:foldmethod=marker:foldmarker=<<<,>>>
\documentclass[compress,aspectratio=169]{beamer}

%<<< Preamble
\usepackage[english]{babel}
\usepackage{metalogo}
\usepackage{listings}
\usepackage{fontspec}
\usepackage{amsmath, amssymb, bm}
\usepackage{stackrel}
\usepackage{tikz}
\usepackage{unicode-math}
\usepackage{subcaption}
\usepackage[theme=nord,charsperline=60,linenumbers]{jlcode}
% \usepackage[charsperline=60,linenumbers]{jlcode}

% \usetheme{Boadilla}
\usetheme{Nord}

\setmainfont{Roboto}
% \setsansfont{DejaVu Serif}
% \setmonofont{CaskaydiaCove Nerd Font Mono}
% \setmonofont{JuliaMono}
\setmonofont{FiraCode Nerd Font Mono}

\setbeamercolor{mybullet}{use=itemize item.fg,bg=itemize item.fg,fg=itemize item.fg}

\makeatletter
\def\verbatim@nolig@list{}
\newcommand\pin{%
\parbox[t]{10pt}{\raisebox{0.2pt}{\usebeamercolor[fg]{mybullet}{$\ast$}}}}
\makeatother

\newcommand{\E}[1]{\ensuremath{E\left\{#1\right\}}}
\newcommand{\norm}[1]{\ensuremath{\lVert#1\rVert}}
\newcommand*{\thead}[1]{\multicolumn{1}{c}{\fseries #1}}

\newfontfamily\tabulartext[SizeFeatures={Size=6}]{Roboto}

\hypersetup{
    colorlinks=true,
    urlcolor=NordBlue
}

\DeclareMathOperator*{\argmax}{argmax}
\DeclareMathOperator*{\Var}{Var}
\DeclareMathOperator*{\test}{\gtrless}

% \AtBeginDocument{
%     \fontsize{8}{12}
%     \selectfont
% }

\AtBeginSection[]
{
    \begin{frame}[c,noframenumbering,plain]
        \tableofcontents[sectionstyle=show/hide,subsectionstyle=show/show/hide]
    \end{frame}
}


\AtBeginSubsection[]
{
    \begin{frame}[c,noframenumbering,plain]
        \tableofcontents[sectionstyle=show/hide,subsectionstyle=show/shaded/hide]
    \end{frame}
}
%>>>

\title{A Tour of Julia}
\subtitle{Future of coding}
\author{\large Simon Andreas Bjørn}
\date{\large March 14, 2023}

\begin{document}

\begin{frame}[plain,noframenumbering] % <<<
    \maketitle
    \begin{columns}
        \begin{column}{0.7\textwidth}
        \end{column}
        \begin{column}{0.3\textwidth}
            \begin{figure}
                \includegraphics[width=\columnwidth]{"julia-logo-color.png"}
            \end{figure}
        \end{column}
    \end{columns}
\end{frame}
% >>>

\begin{frame} % <<< What is Julia
    \frametitle{What is Julia}
    \begin{itemize}
        \item General Purpose
        \item Dynamically Typed
        \item High Performance, JIT
        \item Multi-platform
        \item Numerical Language
    \end{itemize}
\end{frame} % >>>

\begin{frame}[fragile] % <<< What does Julia look like
    \frametitle{What does Julia look like}
    \begin{jllisting}[gobble=8]
        function simulate_launc(spaceship::SpaceVehicle, Δt::Number; max_duration::Number = 2000)
            t = 0  # Start time
            ship = copy(spaceship)
            while whip.active_stage isa Rocket
                while propellant(ship) > 0 && t <= max_duration
                    boosters = sideboosters(ship)
                    if !isempty(boosters) && sum(propellant.(boosters)) <= 0
                        detach_sideboosters!(ship)
                    end
                    update!(ship, t, Δt)
                    t += Δt
                end
                stage_separate!(ship)
            end
        end
    \end{jllisting}
\end{frame} 
% >>>

% Language tour

\begin{frame} % <<< Language Tour :D
    \frametitle{Language Tour :D}

    \begin{itemize}
        \item Functions
        \item Variables
        \item Loops
        \item Conditions
        \item Arrays
    \end{itemize}
\end{frame} % >>>

\begin{frame}[fragile] % <<< Working with Julia
    \frametitle{Working with Julia}
    \begin{columns}
        \begin{column}{0.4\textwidth}
            \begin{itemize}
                \item REPL-based workflow
                \item Modules \& packages
                \item Focus on VSCode integration
                    \begin{itemize}
                        \item In-Editor plots
                        \item Jupyter kernel
                        \item Line \& code-block execution
                    \end{itemize}
            \end{itemize}
        \end{column}
        \begin{column}{0.6\textwidth}
            Simple REPL and package example
            \begin{verbatim}
julia> print("Hello world!")
Hello world!

julia> 2+2
4

julia>]
(@v1.8) pkg> add LinearAlgebra

julia> using LinearAlgebra

julia> norm(randn(2,2))
1.995642503223222
            \end{verbatim}

        \end{column}
    \end{columns}
\end{frame}
% >>>

\begin{frame}[fragile] % <<< Composite Types
    \frametitle{Composite Types}
    Defining and combining types for concrete tasks.

    \begin{columns}
        \begin{column}{0.4\textwidth}
            \begin{itemize}
                \item Different from classes (C++ struct vs. class)
                \item Type-hinting
                \item Imutable by default.
            \end{itemize}
        \end{column}
        \begin{column}{0.6\textwidth}
            Python
            \begin{jllisting}[gobble=16, language=python]
                class Knight:
                    def __init__(self, name, health, armor):
                        self.name = name
                        self.health = health
                        self.armor = armor
            \end{jllisting}
            \begin{columns}
                \begin{column}{0.5\textwidth}
                    C/C++
                    \begin{jllisting}[gobble=24,language=c]
                        struct Knight {
                            string name;
                            int health;
                            int armor;
                        }
                    \end{jllisting}
                \end{column}
                \begin{column}{0.5\textwidth}
                    Julia
                    \begin{jllisting}[gobble=24]
                        struct Knight
                            name::String
                            health::Int
                            armoar::Int
                        end
                    \end{jllisting}
                \end{column}
            \end{columns}
        \end{column}
    \end{columns}
\end{frame}
% >>>

\end{document}
