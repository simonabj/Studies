% !TEX program = xelatex
% vim:foldmethod=marker:foldmarker=<<<,>>>
\documentclass[compress,aspectratio=169]{beamer}

%<<< Preamble
\usepackage[english]{babel}
\usepackage{metalogo}
\usepackage{listings}
\usepackage{fontspec}
\usepackage{amsmath, amssymb, bm}
\usepackage{stackrel}
\usepackage{tikz}
\usepackage{unicode-math}
\usepackage{subcaption}
% \usepackage[theme=nord,charsperline=60,linenumbers]{jlcode}
\usepackage[charsperline=60,linenumbers]{jlcode}

\usetheme{Boadilla}
% \usetheme{Nord}

\setmainfont{Roboto}
% \setsansfont{DejaVu Serif}
% \setmonofont{CaskaydiaCove Nerd Font Mono}
\setmonofont{JuliaMono}

\setbeamercolor{mybullet}{use=itemize item.fg,bg=itemize item.fg,fg=itemize item.fg}

\makeatletter
\def\verbatim@nolig@list{}
\newcommand\pin{%
\parbox[t]{10pt}{\raisebox{0.2pt}{\usebeamercolor[fg]{mybullet}{$\ast$}}}}
\makeatother

\newcommand{\E}[1]{\ensuremath{E\left\{#1\right\}}}
\newcommand{\norm}[1]{\ensuremath{\lVert#1\rVert}}
\newcommand*{\thead}[1]{\multicolumn{1}{c}{\fseries #1}}

\newfontfamily\tabulartext[SizeFeatures={Size=6}]{Roboto}

\hypersetup{
    colorlinks=true,
    urlcolor=NordBlue
}

\DeclareMathOperator*{\argmax}{argmax}
\DeclareMathOperator*{\Var}{Var}
\DeclareMathOperator*{\test}{\gtrless}

\AtBeginDocument{
    \fontsize{8}{12}
    \selectfont

}

\AtBeginSection[]
{
    \begin{frame}[c,noframenumbering,plain]
        \tableofcontents[sectionstyle=show/hide,subsectionstyle=show/show/hide]
    \end{frame}
}


\AtBeginSubsection[]
{
    \begin{frame}[c,noframenumbering,plain]
        \tableofcontents[sectionstyle=show/hide,subsectionstyle=show/shaded/hide]
    \end{frame}
}
%>>>

\title{Project IV: Detection theory}
\subtitle{}
\author{Simon Andreas Bjørn}
\date{March 14, 2023}

\begin{document}

\begin{frame}[plain,noframenumbering]
    \maketitle
\end{frame}

\begin{frame} % <<< PDF of measurement
    \frametitle{PDF of measurement}
    We first want to define the PDF of $f_\theta\left(x\right)$ which is the
    PDF of the {\em signal} + {\em noise}
    \begin{columns}
        \begin{column}{0.5\textwidth}
            We have a SNR-PDF model as shown on the right.
            We notice that:
            \begin{itemize}
                \item Noise PDF is i.i.d. and sampled around 0
                \item Signal PDF $f_\theta$ average at $\theta$
                \item The noise is additive to the signal
            \end{itemize}
        \end{column}
        \begin{column}{0.5\textwidth}
            \begin{figure}
                \includegraphics[width=\columnwidth]{"../PDFs.pdf"}
            \end{figure}
        \end{column}
    \end{columns}
    With the noise characterized and distributed $n \sim \mathcal{N}\left(0,\sigma\right)$, we have that
    the signal is distributed $s \sim \mathcal{N}\left(\theta,\sigma\right)$
    which gives the PDF
    \begin{equation*}
        f_\theta (x) = \frac{1}{\sqrt{2\pi}\sigma^2}e^{-\frac{\left(x-\theta\right)^2}{2\sigma^2}}
    \end{equation*}
\end{frame} % >>>

\begin{frame}[fragile] % <<< Specific detection problem
    \frametitle{Specific detection problem}
    Now we assume that $\theta = 5$ and $\sigma = 5$. Plotting this yields the 
    following plot
    \begin{columns}
        \begin{column}{0.5\textwidth}
            \begin{jllisting}[gobble=16]
                θ = σ = 5
                fn = Normal(0,σ); fs = Normal(θ,σ)
                lines!(fn); lines!(fs)
            \end{jllisting}
        \end{column}
        \begin{column}{0.5\textwidth}
            \begin{figure}
                \includegraphics[width=\columnwidth]{"../b.pdf"}
            \end{figure}
        \end{column}
    \end{columns}
    I would say that this detection problem is quite tricky, because the two
    distributions are quite close together. The overlap of the false alarm and
    the detected signal is large, and so actually classifying what is noise
    and what is a true signal is difficult.
\end{frame} 
% >>>

\begin{frame}[fragile] % <<< CFAR detector with $P_{FA}] = 0.1$
    \frametitle{CFAR detector with $P_{FA} = 0.1$}
    
    We want a threshold $\beta$ such that $P_{FA} = 0.1$, which is given by 
    \begin{equation*}
        \int^{\infty}_{\beta}{f_n(x)dx} = P_n(\beta \le x) = 1-F_n(\beta)
    \end{equation*}
    where $F_n(x)$ is the CDF of the noise PDF. 
    We know that the quantile function 
    $Q: \left[0,1\right] \rightarrow \mathbb{R}$ is defined as the inverse CDF, s.t.
    $Q_n(q) = F^{-1}_n(q)$. This gives the threshold 
    $\beta = Q_n\left(1-P_{FA}\right)$. With $\beta$ given, we can easly find $P_D$ by
    $1-F_s\left(\beta\right)$.

    \begin{columns}
        \begin{column}{0.5\textwidth}
            \begin{jllisting}[gobble=16]
                β = cquantile(fn, 0.1)
                P_fa = ccdf(fn, β)
                P_d  = ccdf(fs, β)
            \end{jllisting}

            As we can see in the plot and evaluations, we we get a threshold at
            $\beta = 6.41$ with a $P_D = 0.39$. That is, there is a 39\% chance
            for a detection.
        \end{column}
        \begin{column}{0.5\textwidth}
            \begin{figure}
                \includegraphics[width=\columnwidth]{"../c.pdf"}
            \end{figure}
        \end{column}
    \end{columns}
\end{frame}
% >>>

\begin{frame}[fragile] % <<< ROC of detector
    \frametitle{ROC of detector}
    We now want to find the ROC of the detector. We do this by evaluating the detector
    at every possible threshold value.
    \begin{columns}
        \begin{column}{0.51\textwidth}
            Create the data as done in assignmnet
            \begin{jllisting}[gobble=16]
                θ = σ = 5
                nData = 1000
                s = θ * (randn(nData,1) .>= 0); # True data
                x = s + σ * randn(size(s))
            \end{jllisting}
            Then evaluate the true ROC and using the data
            \begin{jllisting}[gobble=16]
                # Theoretical ROC
                βs = 20:-0.01:-20
                TPR_true,TFP_true = ccdf.(fs,βs), ccdf.(fn,βs)
                # Emprirical ROC
                detector = map(β -> β .< x, βs)
                TPR_data = map(result -> mean(result[s .> 0.5θ]), detector)
                TFP_data = map(result -> mean(result[s .< 0.5θ]), detector);
            \end{jllisting}
        \end{column}
        \begin{column}{0.49\textwidth}
            By plotting TFP against TPR, we get the following ROC plot
            \begin{figure}
                \includegraphics[width=0.7\columnwidth]{"../d.pdf"}
            \end{figure}
        \end{column}
    \end{columns}
\end{frame} 
% >>>

\begin{frame}[fragile] % <<< Several realizations
    \frametitle{Several realizations}
    We rap the code above into a function \texttt{roc\_realization()} and evaluate the
    ROC for a few realization. 
    \begin{columns}
        \begin{column}{0.6\textwidth}
            \begin{jllisting}[gobble=16]
                many_runs = [ roc_realization() for _ in 1:50 ]
                # Each row is an roc of threshold and columns are each run
                # -> β × run
                TPR_many = mapreduce(run -> run[1], hcat, many_runs)
                TFP_many = mapreduce(run -> run[2], hcat, many_runs)  
                # Takse the mean along runs
                TPR_mean = mean(TPR_many, dims=2) 
                TFP_mean = mean(TFP_many, dims=2);
            \end{jllisting}
            \pin We can see that increasing the number of realizations will
            make the ROC of the data approach the theoretical ROC.
        \end{column}
        \begin{column}{0.4\textwidth}
            \begin{figure}
                \includegraphics[width=\columnwidth]{"../d2.pdf"}
            \end{figure}
        \end{column}
    \end{columns}
\end{frame}
% >>>

% Wrong answer
% \begin{frame} % <<< Multi-measure over transmit
%     \frametitle{Multi-measure over transmit}
%     \begin{columns}
%         \begin{column}{0.6\textwidth}
%             We now want to measure multiple times per transmit, such that 
%             \begin{alignat*}{2}
%                 H_{0}^{'} &: x_n = n_n &&\text{ for } n=1,\dots,N \\
%                 H_{1}^{'} &: x_n = \theta + n_n &&\text{ for } n=1,\dots,N
%             \end{alignat*}
%             We know that the joint PDF for $N$ samples of the white additive noise is
%             given by 
%             \begin{equation*}
%                 f_N\left(n\right) = \frac{1}{\left(2\pi\sigma^2\right)^{N/2}}\exp{\left(
%                         -\sum^{N}_{n=1}{\frac{n_n^2}{2\sigma^2}}
%                 \right) }
%             \end{equation*}
%             Since we see that $x_n$ has a Gaussian distribution, we find the mean
%             $\E{x_n} = \E{x_n} = \E{\theta+n_n}=\theta$.
%
%             Variance is found in a similar manner
%             $\sigma_{x_n}^2=\E{x_n - \E{x_n}}^2 = \E{\theta + n_n - \theta}^2 =
%             \E{n_n}^2=\sigma^2$
%         \end{column}
%         \begin{column}{0.4\textwidth}
%             Each individual PDF is then given by
%             \begin{align*}
%                 f_\theta(x_n) &= \frac{1}{\left(2\pi\sigma^2\right)^{1/2}}\exp\left(-\frac{\left(x_n - \theta\right)^2 }{2\sigma^2}\right) \\
%                 f_n(x_n) &= \frac{1}{\left(2\pi\sigma^2\right)^{1/2}}\exp\left(-\frac{x_n^2 }{2\sigma^2}\right) \\
%             \end{align*}
%             Taking every observation into account, we find the PDFs by the product rule
%             \begin{align*}
%                 f_\theta(\pmb{x}) &= \prod^{N}_{n=1} \frac{1}{\left(2\pi\sigma^2\right)^{1/2}}\exp\left(-\frac{\left(x_n - \theta\right)^2 }{2\sigma^2}\right) \\
%                 f_n(\pmb{x}) &= \prod^{N}_{n=1} \frac{1}{\left(2\pi\sigma^2\right)^{1/2}}\exp\left(-\frac{x_n^2 }{2\sigma^2}\right) \\
%             \end{align*}
%         \end{column}
%     \end{columns}
% \end{frame} % >>>

\begin{frame}[fragile] % <<< Multi-measure over transmit
    \frametitle{Multi-measure over transmit}
    \begin{columns}
        \begin{column}{0.5\textwidth}
            We now want to measure multiple times per transmit, such that 
            \begin{alignat*}{2}
                H_{0}^{'} &: x_n = n_n &&\text{ for } n=1,\dots,N \\
                H_{1}^{'} &: x_n = \theta + n_n &&\text{ for } n=1,\dots,N
            \end{alignat*}
            \pin Both distribution are sampled $N$ times and we take the mean.
            \begin{equation*}
                Z = \frac{1}{N}\sum^{N}_{i=1}{X_i}
            \end{equation*}
            \pin We use the central limit theorem to find the distribution of
            each variable
            \begin{align*}
                H_0^{'} &: z_0      \sim \mathcal{N}\left(0,\sigma / \sqrt{n} \right) \\
                H_1^{'} &: z_\theta \sim  \mathcal{N}\left(\theta, \sigma / \sqrt{n} \right)
            \end{align*} 
            \pin PDFs are Normal distributions with $\sigma_N = \sigma/\sqrt(N)$
        \end{column}
        \begin{column}{0.5\textwidth}
            \begin{jllisting}[gobble=16]
                multi_fn = Normal(0, σ/sqrt(N))
                multi_fs = Normal(θ, σ/sqrt(N))
                β = cquantile(multi_fn, 0.1)
                P_fa, P_d = ccdf(multi_fn, β), ccdf(multi_fs, β)
            \end{jllisting}
            This gives $P_{FA}=0.1$ to be $\beta = 2.42$ which gives a $P_D=0.91$.
            \begin{figure}
                \includegraphics[width=\columnwidth]{"../f.pdf"}
            \end{figure}
        \end{column}
    \end{columns}
\end{frame} 
% >>>

\begin{frame}[fragile] % <<< ROC with multiple measurements
    \frametitle{ROC with multiple measurements}
    We want to find the ROC for different number of measurements $N$ per observation.

    \begin{columns}
        \begin{column}{0.5\textwidth}

            Define some functions, and loop over each value of $N$
            \begin{jllisting}[gobble=16]
                function make_observations(signal, noise_pdf, N)
                    S = repeat(signal, 1, N)
                    X = S .+ rand(noise_pdf, size(S))
                    return X
                end

                mean_detector(observations, threshold) = mean(observations, dims=2) .>= threshold
            \end{jllisting}
            Then iterate over each value of $N$
            \begin{jllisting}[gobble=16]
                for (i,n) in enumerate([1,2,3,7,14])
                    noise_pdf, signal_pdf = mean_cfar(fn, fs, n)
                    X = make_observations(signal, fn, n)
                    detector = map(β -> mean_detector(X, β), βs)
                    # ...
                end
            \end{jllisting}
        \end{column}
        \begin{column}{0.5\textwidth}
            Plotting the ROC for each detector, yields the following plot
            \pin Increasing $N$ gives a better detector

            \pin Data follows true roc as expected
            \begin{figure}
                \includegraphics[width=0.8\columnwidth]{"../g.pdf"}
            \end{figure}
        \end{column}
    \end{columns}
    
\end{frame}
% >>>

\begin{frame} % <<< Detector problem when $P_0 \ne P_\theta$
    \frametitle{Detector problem when $P_0 \ne P_\theta$}
    We now want to look at a detection problem where the ratio between $0$ and
    $\theta$ differ from $1$.

    \begin{columns}
        \begin{column}{0.5\textwidth}

            \pin We find the ROC for this detector problem
            \begin{figure}
                \includegraphics[width=0.7\columnwidth]{"../h.pdf"}
            \end{figure}
        \end{column}
        \begin{column}{0.5\textwidth}
            \pin There is no difference between the ROC plot when assuming 
            $P_0 = P_\theta$ and with $P_0 \ne P_\theta$
            \pin When assuming equal probability $P_0 = P_1$, ignoring cost gives
            the maximum likelihood (ML) criterion
            \begin{equation*}
                \frac{f_\theta(\pmb{y})}{f_0(\pmb{y})} \test^{H_1}_{H_0} 1
            \end{equation*}
            \pin With {\em a priori} probability $P_0 \ne P_1$, ignoring cost gives 
            the maximum {\em a posteriori} (MAP) criterion
            \begin{equation*}
                \frac{f_\theta(\pmb{y})}{f_0(\pmb{y})} \test^{H_1}_{H_0} \frac{P_0}{P_1}
            \end{equation*}
            \pin Only difference is threshold value given observations
        \end{column}
    \end{columns}

\end{frame} % >>>

\end{document}
