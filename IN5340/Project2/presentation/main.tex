% !TEX program = xelatex
% vim:foldmethod=marker:foldmarker=<<<,>>>
\documentclass[compress]{beamer}

%<<< Preamble
\usepackage[english]{babel}
\usepackage{metalogo}
\usepackage{listings}
\usepackage{fontspec}
\usepackage{amsmath, amssymb}
\usepackage{stackrel}
\usepackage{tikz}
\usepackage{svg}
\usepackage{unicode-math}
\usepackage{subcaption}
\usepackage[theme=nord,charsperline=60,linenumbers]{jlcode}

\usetheme{Nord}

\setmainfont{Roboto}
% \setsansfont{DejaVu Serif}
% \setmonofont{CaskaydiaCove Nerd Font Mono}
\setmonofont{JuliaMono}

\makeatletter
\def\verbatim@nolig@list{}
\makeatother

\newcommand{\E}[1]{\ensuremath{E\left\{#1\right\}}}

\hypersetup{
    colorlinks=true,
    urlcolor=NordBlue
}

\DeclareMathOperator*{\argmax}{argmax}

\AtBeginDocument{
    \fontsize{8}{12}
    \selectfont

}

\AtBeginSection[]
{
    \begin{frame}[c,noframenumbering,plain]
        \tableofcontents[sectionstyle=show/hide,subsectionstyle=show/show/hide]
    \end{frame}
}


\AtBeginSubsection[]
{
    \begin{frame}[c,noframenumbering,plain]
        \tableofcontents[sectionstyle=show/hide,subsectionstyle=show/shaded/hide]
    \end{frame}
}
%>>>

\title{Project II: Estimation Theory}
\subtitle{Maximum likelihood, CRLB \& Least Squares Estimation}
\author{\Large Simon Andreas Bjørn}
\date{\large February 28, 2023}

\begin{document}

\begin{frame}[plain,noframenumbering]
    \maketitle
\end{frame}

\begin{frame} % <<< 1A.1
    \frametitle{1A - Maximum Likelihood Estimator}
    We want to derive the maximum likelihod esimator for the time delay of a pulse
    in additive gaussian noise. The sent pulse $s(t)$ is received as $x(t)$ after
    the time delay $\tau_0 = 2r/c$ which is what we want to estimate.
    \medskip
    \begin{columns}[t]
        \begin{column}{0.4\textwidth}
            \centerline{Model for continues signal}
            \begin{equation*}
                x(t) = s(t-\tau_0) + W(t)
            \end{equation*}

            \begin{itemize}
                \item $W(t)$ = AGN
                \item $s(t-\tau_0)$ = received signal. 
                \item Assume $W$ is bandlimited by $B_w$.
                \item Sample $W(t)$ with $\Delta=\frac{1}{2}B_w$ s.t. $W[n]=W(n\Delta)$.
                \item Sampeling the entire noise bandwidth, we can assume the noise to be
                white. 
            \end{itemize}
        \end{column}
        \begin{column}{0.7\textwidth}
            \centerline{Model for discrete signal}
            \begin{equation*}
                x[n] = s[n-n_0] + W[n]
            \end{equation*}
            \begin{itemize}
                \item Discrete time-delay is given by $\tau_0=n_0\Delta$
                \item $s[n]$ is non-zero only during pulse length: \\
                    \begin{equation*}
                        x[n] = 
                        \begin{cases}
                            W[n] & , 0 \le n \le n_0-1 \\
                            s[n-n_0]+W[n] & , n_0 \le n < n_0 + M \\
                            W[n] & , n_0 + M \le n \le N
                        \end{cases}
                    \end{equation*}
            \end{itemize}
        \end{column}
    \end{columns}
\end{frame} %>>>

\begin{frame} %<<< 1A.2
    \frametitle{1A - Maximum Likelihood Estimator 2}
    White gaussian noise means $W[n]$ is independent, so the PDF of $s[n;n_0]$
    can be factorized into the PDFs of each sub-interval.
    Split into 3:
    \begin{enumerate}
        \item $x[n]=W[n],\; 0 \le n \le n_0-1$:
            \begin{equation*}
                \prod^{n_0-1}_{n=0}{\frac{1}{\sqrt{2\pi\sigma^2}}
                \exp\left(-\frac{x^2[n]}{2 \sigma^2}\right)}
            \end{equation*}
        \item $x[n]=S\left[n-n_0\right]+W[n], \; n_0 \le n \le n_0+M-1$:
            \begin{equation*}
                \prod^{n_0+M-1}_{n=n_0}{\frac{1}{\sqrt(2\pi\sigma^2)}
                \exp\left(-\frac{\left(x[n]-s[n-n_0]\right)^2}{2\sigma^2}\right)}
            \end{equation*}
        \item $x[n] = W[n], \; n_0+M \le n \le N_1$:
            \begin{equation*}
                \prod^{N-1}_{n=n_0+M}{\frac{1}{\sqrt{2\pi\sigma^2}}
                \exp\left(-\frac{x^2[n]}{2\sigma^2}\right)}
            \end{equation*}
    \end{enumerate}
\end{frame} %>>>

\begin{frame} %<<< 1A.3
    \frametitle{1A - Maximum Likelihood Estimator 3}
    We can expand and factorize 1., 2. \& 3. 
    \begin{gather*}
        P(x; n_0)  = \left(\frac{1}{\sqrt{2\pi\sigma^2}}\right)^{N-1}
        \exp\left(-\frac{\sum^{N-1}_{n=0}{x^2[n]}}{2\sigma^2}\right) \\
        \cdot\exp\left(
            -\frac{1}{2\sigma^2}
            \underbrace{
                \sum^{n_0+M-1}_{n=n_0}{\left(
                        -2x[n]s[n-n_0]+s^2[n-n_0]
                \right)}
            }_{\text{We want to maximize this}}
        \right)
    \end{gather*}
    So we have 
    \begin{equation*}
        \underbrace{2}_{\text{const.}}
        \cdot
        \underbrace{
            \sum^{n_0+M-1}_{n=n_0}{x[n]s[n-n_0]}
        }_{\text{maximize}}
        +
        \underbrace{
            \sum^{n_0+M-1}_{n=n_0}{s^2[n-n_0]}
        }_{\text{not dependent of } n_0}
    \end{equation*}
\end{frame} %>>>

\begin{frame} % <<< Maximum Likelihood Estimator 4
  \frametitle{Maximum Likelihood Estimator 4}
    $s[n]$ is 0 outside of limits so we can expand the sum.
    \begin{equation*}
        \argmax_{n_0}
        \left\{
        \sum^{n_0+M-1}_{n=n_0}{x[n]s[n-n_0]}
        \right\} = 
        \argmax_{n_0}\left\{
            \underbrace{ \sum^{N-1}_{n=0}{x[n]s[n-n_0]} }_{
                \text{This is just a cross correlation}
            }
        \right\}
    \end{equation*}
    We then end up with the Maximum Likelihood Estimator (MLE) to be the
    maximum value of the cross correlation of the signal
    \begin{equation*}
    \widehat{n_0}=\argmax_{0\le m \le N-M}\left\{C_{xs}\left[m\right]\right\}
    \end{equation*}
\end{frame} % >>>

\begin{frame}[fragile] % <<< Implementing the ML estimator
    \frametitle{Implementing the ML estimator}
    We now want to implement the ML esimator.

    Note: Unnecessary code is left out. (Styling and so on). If curious,
    check repo at \href{https://github.com/Dainou01/Studies/tree/main/IN5340/Project2}{here}
    \begin{jllisting}[gobble=8]
        #  Start by getting some packages for this project
        using DSP; using Plots; using Unitful; using MAT
        import Unitful: Time, s, m, Hz
        default(background_color=:transparent, foreground_color=:white)

        # Import the MAT file 
        mat_file = matopen("sonardata2.mat")

        B    = read(mat_file, "B")    * Hz
        fs   = read(mat_file, "fs")   * Hz
        c    = read(mat_file, "c")    * m/s
        T_p  = read(mat_file, "T_p")  * s
        fc   = read(mat_file, "fc")   * Hz
        t_0  = read(mat_file, "t_0")  * s
        data = read(mat_file, "data")
    \end{jllisting}
\end{frame} 
% >>>

\begin{frame}[fragile] % <<< Implementing the ML estimator 2
    \frametitle{Implementing the ML estimator 2}
    Now we can define the signal and plot it over time
    \begin{columns}
        \begin{column}{0.5\textwidth}
            \begin{jllisting}[gobble=16]
                # Signal model
                α = B / T_p

                s_Tx(t::Time) = (-T_p/2<=t<=T_p/2) 
                    ? exp(2π*α*t^2/2*im) : 0.0im
                # Convert any input to Time
                s_Tx(t) = s_Tx(t*s) 

                # Plot simple signal
                t = -T_p/2 : 1/fs : T_p/2
                plot(t,real.(s_Tx.(t)))
                plot!(t, imag.(s_Tx.(t)))
            \end{jllisting}
        \end{column}
        \begin{column}{0.5\textwidth}
            Plotted output
            \begin{figure}
                \includesvg[width=\columnwidth]{"../1b.svg"}
            \end{figure}
        \end{column}
    \end{columns}
\end{frame}
% >>>

\begin{frame}[fragile] % <<< Implementing the ML estimator 3
    \frametitle{Implementing the ML estimator 3}
    Now we implement the xcorr operation...
    \begin{jllisting}[gobble=8]
        match_filter(ch,signal) = xcorr(ch, signal, padmode=:longest)[end÷2+1:end]
    \end{jllisting}
    and we test it by applying it to 1 channel and the source signal
    \begin{columns}
        \begin{column}{0.53\textwidth}
            \begin{jllisting}[gobble=16]
                ml₁ = match_filter(data[:, 1], s_Tx.(t))
                plot(
                    axes(ml₁,1)/upreferred(fs),
                    abs.(ml₁),
                )
            \end{jllisting}
        \end{column}
        \begin{column}{0.47\textwidth}
            \begin{figure}
                Plot generated from code
                \includesvg[width=\columnwidth]{"../1b2.svg"}
            \end{figure}
        \end{column}
    \end{columns}
\end{frame}
% >>>

\begin{frame}[fragile] % <<< Applying the ML estimator
    \frametitle{Applying the ML estimator}
    Now we map this as a function on each channel and plot the results.
    \begin{jllisting}[gobble=8]
        compressed = mapslices(
            ch -> match_filter(ch, s_Tx.(t)),
            data[1:1600,:], dims=(1,)) 

        heatmap(axes(compressed,2), axes(compressed,1)/upreferred(fs), 
            amp2db.(abs.(compressed)/maximum(abs.(compressed))))
        plot([
            plot((1:1600)/upreferred(fs), abs.(d[1:1600,1]))
            for d in [data, compressed]
        ]..., layout=[1,1],)
    \end{jllisting}
    \begin{figure}
        \centering
        \begin{subfigure}{0.5\textwidth}
            \centering
            \includesvg[width=\textwidth]{"../1c.svg"}
        \end{subfigure}%
        \begin{subfigure}{0.5\textwidth}
            \centering
            \includesvg[width=\columnwidth]{"../1c2.svg"}
        \end{subfigure}
    \end{figure}
\end{frame}
% >>>

\begin{frame} % <<< Applying the ML estimator 2
    \frametitle{Applying the ML estimator 2}
    \begin{itemize}
        \item Looking at the 1600 samples of the raw data, we see that it looksquite noisy,
            but there seems to be some kind of signal around 0.01s. 

        \item When we perform the cross correlation, we see that a strong peak appear at
            about 5ms. By the ML estimator this is when it is most likely when we
            recieve the pulse.
    \end{itemize}
\end{frame} % >>>




\end{document}
